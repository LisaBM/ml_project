This is the largest programming project any one of has ever worked on and neither of us had done much with Python before or had any other machine learning experience. So at any given time there was always a great variety of difficulties at hand. And because we implemented everything ourselves the problem could be literally anywhere, from misunderstanding the pseudocode we used, over float and integer division or git merging issues to ... But because that every little success was a source of joy which grew with every working block of code we managed to compile.
\\
\\
The overall structure of our program looks as follows...(sorry havent had time yet) 
\\
I couldn't excatly find where we do cross validation? would have only needed it if we had less training data?
\\
\\
We trained our SVMs with a linear and a Gaussian kernel and with different numbers training points. Our given data set included $42 000$ training points. We never used all of them for training because it simply exceeded our available compute power. But from the table below we can see that it would probably not have improved our results considerably or might even have worsened them. And this way we always had engouh labeled data to test and verify our results with. In the table you can find the results of the different multi-class classifiers. For the One-vs-All approach we give numbers for how many labels were found correctly with and without using additional classification by location of the barycenters, so that one can really see how many uniquely classified data points are labeled correctly. The percentages are derived from always testing with $1000$ data points that were of course not used for training beforehand. Unsurprisingly ECOC had longer run times than the One-vs-All classifier, it has to train $15$ instead of only $10$ and the computation with the Gaussian kernel took longer than with the linear kernel, which was also to be expected because of its higher complexity (more detail?). For $10 000$ training points ECOC with linear kernel took $1$h $16$ min and ECOC with gaussian kernel took $2$ h $26$ min for training.  

As expected from theoretical considerations the Error-Correcting Output Codes performed better than the One-vs-All classification for training sets of any size. However we were surprised how well One-vs-All with the gaussian kernel worked in the end. Every picture of a hand-written digit gets turned into a vector of size $784$ ($=28^2$), it was not clear that the output data would be anywhere close to being linearly separable so we were also suprised to see the linear classifiers working at all, especially in the first case were not a single ambiguity or mistake could be corrected. 

How do we explain the decrese?
 
ECOC linear 10000 points: 1 h 16 min
ECOC Gauss 10000 points: 2 h 26 min
So we can say at least how long the longest jobs took. Is that, combined with an extra benchmark section only focused on the comparison of the 5 SMOs applied to only the first of the  15 ECOC classifiers, both kernels, enough? 
----> Is what wrote enough otherwise please add on :) 

% Ich hätte ganz links oben gerne eine schräge Linie
% \lft{n}\rt{k}}
% \lft{classifier}\rt{no of training points}	
\begin{table}[ht!]
	\centering
	\caption{Overview Results}
	\begin{tabular}{|l|l|l|l|l|l|l|l|l|l|l|l|} \hline
\multicolumn{1}{|p{2cm}|}{no of training points classifier} & \multicolumn{1}{p{2cm}|}{One-vs-All uniquely classfied correctly linear} & \multicolumn{1}{p{2cm}|}{One-vs-All w/ barycenters linear} & \multicolumn{1}{p{2cm}|}{One-vs-All uniquely classfied correctly Gaussian} &  \multicolumn{1}{p{2cm}|}{One-vs-All w/ barycenters Gaussian} & \multicolumn{1}{p{2cm}|}{ECOC linear} & \multicolumn{1}{p{2cm}|}{ECOC Gaussian} \\ \hline \hline
500	& 65.9\% & 74.1\% & 75.4\% & 83.3\% & 74.2\% & 87.4\% \\ \hline
1000	& 68.2\% & 75.0\% & 84.3\% & 89.0\% & 78.0\% & 92.7\% \\ \hline
2000	& 70.2\% & 76.4\% & 89.8\% & 91.9\% & 77.8\% & 94.3\% \\ \hline
5000	& 70.0\% & 73.8\% & 889\% & 91.6\% & 82.0\% & 95.2\% \\ \hline
10000	& 64.6\% & 67.5 & 88.0\% & 90,6\% & 82.5\% & 95.4\% \\ \hline
	\end{tabular}
\end{table}

We have learned a lot in the past 4 weeks and although this program does not (yet?) win us a kaggle competition we are in general (very) satisfied with the overall outcome of the project and especially with our success rate of over $95$\%.
