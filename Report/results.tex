When choosing ... many training points and .... many testing points our algorithm was correct in ... of the cases. Include things here...?


Linear and Gaussian.

In general ECOC?

Where do we bring in cross validation?

General Conclusions? Difficulties?

For One-vs-All linear almost (<5 \%) no correct result, barycenters improved things a lot  (maybe take out of table if its too bad...)

One-vs-All Gauss: a lot better, I think especially more training data was helpful

ECOC 

Table for overview of results 

Each time we tested with ... 1000 (!!!!!!!!!!!) test points. The given training data set that we had was very large, so we always had data to test our results that was not used for training. 

Talk about average compute times. Or did Paul already say something about that beforehand? 

----->
Sorry, haven't really tracked compute times consistently. Although I can say this much: 
ECOC linear 10000 points: 1 h 16 min
ECOC Gauss 10000 points: 2 h 26 min
So we can say at least how long the longest jobs took. Is that, combined with an extra benchmark section only focused on the comparison of the 5 SMOs applied to only the first of the  15 ECOC classifiers, both kernels, enough?


Talk about the 1500 issue or better not?
----> No. I resolved the issue and I suppose it was a strange coincidence that the problem occurred specifically at 1500 training points. Will talk about the cause of this problem and its resolution in the SMO section.


% Ich hätte ganz links oben gerne eine schräge Linie
% \lft{n}\rt{k}}
% \lft{classifier}\rt{no of training points}	
\begin{table}[ht!]
	\centering
	\caption{Overview Results}
	\begin{tabular}{|l|l|l|l|l|l|l|l|l|l|l|l|} \hline
\multicolumn{1}{|p{2cm}|}{no of training points classifier} & \multicolumn{1}{p{2cm}|}{One-vs-All uniquely classfied correctly linear} & \multicolumn{1}{p{2cm}|}{One-vs-All w/ barycenters linear} & \multicolumn{1}{p{2cm}|}{One-vs-All uniquely classfied correctly Gaussian} &  \multicolumn{1}{p{2cm}|}{One-vs-All w/ barycenters Gaussian} & \multicolumn{1}{p{2cm}|}{ECOC linear} & \multicolumn{1}{p{2cm}|}{ECOC Gaussian} \\ \hline \hline
	500	& 659 & 741 & 754 & 833 & 747 & 874 \\ \hline
	1000	& 682 & 750 & 843 & 890 & 787 & 927 \\ \hline
	2000	& 702 & 764 & 898 & 919 & 778 & 943 \\ \hline
	5000	& 700 & 738 & 889 & 916 & 820 & 952 \\ \hline
	10000	& 646 & 675 & 880 & 906 & 825 & 954 \\ \hline
	\end{tabular}
\end{table}

