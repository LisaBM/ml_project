%!TEX program = xelatex
\documentclass[10pt, compress]{beamer}
\usetheme{Szeged}
\usecolortheme{beaver}

\usepackage{booktabs}
\usepackage[scale=2]{ccicons}
%\usepackage{minted}
%\usepackage{comment}
%\usepgfplotslibrary{dateplot}
%\usepackage[main=german]{babel}  
%\usemintedstyle{trac}
\usepackage[utf8]{inputenc}
%\usepackage[sfdefault]{ClearSans} %% option 'sfdefault' activates Clear Sans as the default text font
\usepackage[T1]{fontenc}


\title{Digit Recognition}
\subtitle{with Support Vector Machines}
\date{\today}
\author{Lisa \textsc{Gaedke-Merzh{\"a}user} \\% Your name
			Paul \textsc{Korsmeier}\\
			Lisa \textsc{Mattrisch}\\
			Vanessa \textsc{Schreck}\\}
\institute{Freie Universit{\"a}t Berlin, Mathematical Aspects of Machine Learning}


\begin{document}

\noindent

\maketitle

\begin{frame}[fragile]
  \frametitle{Overview}

\textbf{Outline} 
  \begin{enumerate}
        \item Introduction \& Problem Statement \item Support Vector Machines (SVM) \item Sequential Minimal Optimization (SMO) \item Multi-Class Classification \item Results \& Conclusions
      \end{enumerate}


%  The \emph{mtheme} is a Beamer theme with minimal visual noise inspired by the
%  \href{https://github.com/hsrmbeamertheme/hsrmbeamertheme}{\textsc{hsrm} Beamer
%  Theme} by Benjamin Weiss.

 % Enable the theme by loading

%  \begin{minted}[fontsize=\small]{latex}
%    \documentclass{beamer}
%    \usetheme{m}
%  \end{minted}

%  Note, that you have to have Mozilla's \emph{Fira Sans} font and XeTeX
%  installed to enjoy this wonderful typography.
\end{frame}


\begin{frame}[fragile]
  \frametitle{Introduction \& Problem Statement}
  \textbf{Idee}
      \begin{itemize}
        \item Existenz von fair handelnden Individuen blablablablabla \item Soziale Ziele sind nicht für alle Menschen gleichgültig \item Infragestellung der Annahmen im Standardmodellvvvvvvvvvv
      \end{itemize}
  %\begin{minted}[fontsize=\small]
  %{latex}
  %  Es gibt nicht nur egoistische Menschen, 
  %  wie es die meisten ökonomischen Modelle annehmen, 
  %  sondern auch solche die fair handeln
  %  und denen soziale Ziele nicht gleichgültig sind
  %\end{minted}
   
   \textbf{Motivation} 
   
   \begin{enumerate}
        \item Handelsmacht wird  in Wettbewerbssituationen ausgenutzt,
        in biliteralen Situationen nicht. 
        \item Trittbrettfahrerei wird in freiwilligen Kooperationsspielen ausgenutzt.
        Besteht allerdings die Möglichkeit, Trittbrettfahrer zu bestrafen, wird diese 
        wahrgenommen auch wenn es kostspielig ist.
      \end{enumerate}
   

   
   
%  for which the \emph{mtheme} provides a nice progress indicator \ldots
\end{frame}




\begin{frame}[fragile]
  \frametitle{Introduction \& Problem Statement}


\textbf{\alert{Matthew Rabin (1993):}}
\begin{itemize}
\item People like to help those who are helping them, and to hurt those who are hurting them
\item Reziprozität
\end{itemize}

\textbf{\alert{David K. Levine (1998):} }
\begin{itemize}

\item Menschen sind zu einem gewissen Grad altruistisch oder gehässig
\end{itemize}

\textbf{\alert{Gary E. Bolton \& Axel Ockenfels (2000):}}
\begin{itemize}
 \item Ähnlich wie Modell von FS(1999) basierend auf Ungleichheitsaversion 

\end{itemize}

\end{frame}

\begin{frame}[fragile]
\frametitle{Support Vector Machines (SVM)}

\begin{center}
\textbf{Was ist fair?}
\end{center}

 \begin{enumerate}
        \item 2 Arten von Menschen: Egoisten \& Ungleichaverse (faire) Menschen
        \item \emph{n} Spieler mit \emph{i}     \(\in \lbrace 1,...,n \rbrace\)
        \item Vektor der monetären Auszahlungen:  \(x=x_1...,x_n\)       
        \item Nutzenfunktion des Spielers \(i \in \lbrace 1,...,n \rbrace\) 
       
       \end{enumerate}
\end{frame}


\begin{frame}[fragile]
\frametitle{Support Vector Machines (SVM)}

\begin{large}
\begin{center} \(U_i(x) = x_i - \alpha_i \frac{1}{n-1} \sum\limits_{j\neq i}max \lbrace x_j - x_i, 0\rbrace - \beta_i \frac{1}{n-1} \sum\limits_{j\neq i} max \lbrace x_i - x_j, 0 \rbrace\)\end{center}    
\end{large}

\begin{center} es gilt: \(\beta_i \leq \alpha_i\ \)  \&  \( 0\leq\beta_i<1\) \end{center}

\end{frame}


\begin{frame}[fragile]
\frametitle{Sequential Minimal Optimization (SMO)}

\textbf{Ablauf \& Annahmen}
   \begin{itemize}
        \item 2 Spieler (Proposer \& Responder) handeln um die Aufteilung eines festen Betrags (=1) \item Proposer kann dem Responder einen Anteil (share) s vorschlagen mit \(s \in [0,1]\) \item Akzeptanz: Proposer: \(s^P=1-s\) \&  Responder: \(s^R=s\) \item Ablehnung: Beide Spieler erhalten 0 \item Proposer = Spieler 1 \& Responder = Spieler 2
      \end{itemize}



\end{frame}


\begin{frame}[fragile]
\frametitle{Multi-Class Classification}

\textbf{Standard Modell}
\begin{itemize}
\item SVMs are binary classifiers but we needed to be able to differentiate among 10 classes
\item there are different ways to tackle this problem, we decided to mainly focus on two different approaches: 
\end{itemize}
\begin{enumerate}
\item One-vs-All 
\item Error Correcting Output Codes
\end{enumerate}

\end{frame}

\begin{frame}[fragile]
\frametitle{Multi-Class Classification}

\textbf{\alert{1. One-Vs-All}}
\begin{itemize}
\item Auszahlungsdifferenz: \(x_1-x_2=\Pi_1 - \Pi_2 = (1-s) -s = {1-2s}\)

\item Nutzenverlust durch Auszahlungsdifferenz für Spieler 2:  \(\alpha_2 (1-2s)\)

\item Akzeptanz wenn \begin{large}\(s-\alpha_2 (1-2s) \geq 0 \Leftrightarrow \frac{s}{1-2s} \geq \alpha_2\)\end{large}

\item Kleinster akzeptierter Anteil s*=\begin{Large} \(\frac{\alpha_2}{(1+2\alpha_2)}\)\end{Large} \(=s'(\alpha)\) % das ist die Annahmeschwelle/Akzeptanzschwelle


\end{itemize} %hier kann ich die rechnung von Nikolas vorrechnen wennn jemand fragt

\end{frame}

\begin{frame}[fragile]
\frametitle{Multi-Class Classification}

%\textbf{These 1:
%}
\textbf{\alert{2. Error Correcting Output Codes}} 
\textbf{Idea: }

\begin{table}[ht!]
	\centering
	\label{Codewords}
	\begin{tabular}{|l|l|l|l|l|l|l|l|l|l|l|l|l|l|l|l|}
		\hline
		Class	& $f_0$ & $f_1$ & $f_2$ & $f_3$ & $f_4$ & $f_5$ & $f_6$ & $f_7$ & $f_8$ & $f_9$ & $f_{10}$ & $f_{11}$ & $f_{12}$ & $f_{13}$ & $f_{14}$ \\ \hline \hline
		0	& 1 & 1 & -1 & -1 & -1 & -1 & 1 & -1 & 1 & -1 & -1 & 1 & 1 & -1 & 1 \\ \hline
		1	& -1 & -1 & 1 & 1 & 1 & 1 & -1 & 1 & -1 & 1 & 1 & -1 & -1 & 1 & -1 \\ \hline
		2	& 1 & -1 & -1 & 1 & -1 & -1 & -1 & 1 & 1 & 1 & 1 & -1 & 1 & -1 & 1 \\ \hline
		3	& -1 & -1 & 1 & 1 & -1 & 1 & 1 & 1 & -1 & -1 & -1 & -1 & 1 & -1 & 1 \\ \hline
		4	& 1 & 1 & 1 & -1 & 1 & -1 & 1 & 1 & -1 & -1 & 1 & 1 & -1 & -1 & 1 \\ \hline
		5	& -1 & 1 & -1 & -1 & 1 & 1 & -1 & -1 & 1 & 1 & -1 & -1 & -1 & -1 & 1 \\ \hline
		6	& 1 & -1 & 1 & 1 & 1 & -1 & -1 & -1 & -1 & 1 & -1 & 1 & -1 & -1 & 1 \\ \hline
		7	& -1 & -1 & -1 & 1 & 1 & 1 & 1 & -1 & 1 & -1 & 1 & 1 & -1 & -1 & 1 \\ \hline
		8	& 1 & 1 & -1 & 1 & -1 & 1 & 1 & -1 & -1 & 1 & -1 & -1 & -1 & 1 & 1 \\ \hline
		9	& -1 & 1 & 1 & 1 & -1 & -1 & -1 & -1 & 1 & -1 & 1 & -1 & -1 & 1 & 1 \\ \hline
	\end{tabular}
\end{table}  

\end{frame}



\end{document}

