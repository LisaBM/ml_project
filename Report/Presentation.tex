%!TEX program = xelatex
\documentclass[12pt, compress]{beamer}
\usetheme{Szeged}
\usecolortheme{beaver}

\usepackage{booktabs}
\usepackage[scale=2]{ccicons}
\usepackage[utf8]{inputenc}

%Font1:
%\usepackage{sansmathfonts}
%\usepackage[T1]{fontenc}
%\renewcommand*\familydefault{\sfdefault}
%Font2:
\usepackage[light,condensed,math]{kurier}
\usepackage[T1]{fontenc}

\usepackage{graphicx}

\newcommand{\titleA}{Introduction to Our Data Set}
\newcommand{\titleB}{Our Approach}
\newcommand{\titleC}{Sequential Minimal Optimization (SMO)}
\newcommand{\titleD}{Multi-Class Classification}
\newcommand{\titleE}{Results \& Conclusions}



\title{Digit Recognition}
\subtitle{with Support Vector Machines}
\date{\today}
\author{Lisa \textsc{Gaedke-Merzh{\"a}user} \\% Your name
			Paul \textsc{Korsmeier}\\
			Lisa \textsc{Mattrisch}\\
			Vanessa \textsc{Schreck}\\}
\institute{Freie Universit{\"a}t Berlin, Mathematical Aspects of Machine Learning}


\begin{document}
\maketitle

\begin{frame}
  \frametitle{Outline} 
  \begin{enumerate}
	  \item \titleA
	  \item \titleB
	  \item \titleC
	  \item \titleD
	  \item \titleE
  \end{enumerate}
\end{frame}


\begin{frame}
  \frametitle{\titleA}
	\textbf{\alert{Main Goal:}} train algorithm to recognize handwritten digits
	\begin{figure}[h]
		\includegraphics[width=1\textwidth]{Digits2}
		%\caption{Visualization of eight of the data points}
	\end{figure}
	\textbf{\alert{Data:}}
	\begin{itemize}
		\item 42,000 greyscale images
		\item 28 by 28 pixels each
		\item partitioned into ten classes
	\end{itemize}
\end{frame}


\begin{frame}
	\frametitle{\titleB}
	We want to use the concept of SVMs
	\begin{itemize}
		\item \textbf{Problem I:} SVMs are binary classifiers $\leftrightarrow$ we have 10 classes
		\item We focus on two different approaches: 
		\begin{enumerate}
			\item One-vs-All 
			\item Error-Correcting Output Codes
		\end{enumerate}
		\item \textbf{Problem II:} Need to solve optimization problem
	\end{itemize}

\end{frame}


\begin{frame}
	\frametitle{\titleB}
	\begin{enumerate}
		\item Implement solver for our QP
		\begin{itemize}
			\item 3 versions
		\end{itemize}
		\item Implement basic SVM algorithm
		\begin{itemize}
			\item linear kernel / Gaussian kernel
			\item Parameter optimization
		\end{itemize}
		\item Combine individual SVMs in different ways
		\begin{itemize}
			\item 3 versions
		\end{itemize}
		\item Validate and compare results 
		\
	\end{enumerate}
\end{frame}



\begin{frame}
	\frametitle{\titleB}

	\begin{figure}[h]
		\includegraphics[width=1\textwidth]{../Plots_linear_mit_Pauls_SVM}
		\caption{Visualizing the concept of linear soft-margin SVMs}
	\end{figure}


\end{frame}


\begin{frame}
	\frametitle{\titleB}
	\begin{enumerate}
		\item Implement solver for our QP
		\begin{itemize}
			\item 3 versions
		\end{itemize}
		\item Implement basic SVM algorithm
		\begin{itemize}
			\item linear kernel / Gaussian kernel
			\item Parameter optimization
		\end{itemize}
		\item Combine individual SVMs in different ways
		\begin{itemize}
			\item 3 versions
		\end{itemize}
		\item Validate and compare results 
		\
	\end{enumerate}
\end{frame}


\begin{frame}
  \frametitle{\titleC}

\end{frame}


\begin{frame}
  \frametitle{\titleD}
	\begin{itemize}
		\item Choose $k$ groups of the classes
		\item Train $k$ SVMs that separate each group from the rest
		\item Compare outcome to what would arise for each digit.
		\item \textbf{\alert{Problem:}} Points may not be classified uniquely.
		\item Handle overlappings by minimizing distance to barycenters
	\end{itemize}

	\begin{figure}[h]
		\includegraphics[width=.6\textwidth]{onevsall_examplegraphic}
		%\caption{Visualization of the One-vs-All approach}
	\end{figure}
\end{frame}

\begin{frame}
  \frametitle{\titleD}
	\textbf{\alert{1. One-vs-All}}\\
	\textbf{Idea:} For each $i \in \{0,1,\ldots,9\}$, train an SVM that separates class $i$ from the rest
  	\begin{table}[ht!]
		\centering
		\scalebox{0.8}{\input{tabular_OneVsAll}}
	\end{table} 
\end{frame}



\begin{frame}
  \frametitle{\titleD}
	\textbf{\alert{2. Error Correcting Output Codes}}\\
	\textbf{Idea:} Relabeling with large Hamming distance according to:
	\begin{table}[ht!]
		\centering
		\scalebox{0.8}{\begin{tabular}{|c|l|l|l|l|l|l|l|l|l|l|l|l|l|l|l|}
	\hline
	Class	& $f_0$ & $f_1$ & $f_2$ & $f_3$ & $f_4$ & $f_5$ & $f_6$ & $f_7$ & $f_8$ & $f_9$ & $f_{10}$ & $f_{11}$ & $f_{12}$ & $f_{13}$ & $f_{14}$ \\ \hline \hline
	0	& 1 & 1 & -1 & -1 & -1 & -1 & 1 & -1 & 1 & -1 & -1 & 1 & 1 & -1 & 1 \\ \hline
	1	& -1 & -1 & 1 & 1 & 1 & 1 & -1 & 1 & -1 & 1 & 1 & -1 & -1 & 1 & -1 \\ \hline
	2	& 1 & -1 & -1 & 1 & -1 & -1 & -1 & 1 & 1 & 1 & 1 & -1 & 1 & -1 & 1 \\ \hline
	3	& -1 & -1 & 1 & 1 & -1 & 1 & 1 & 1 & -1 & -1 & -1 & -1 & 1 & -1 & 1 \\ \hline
	4	& 1 & 1 & 1 & -1 & 1 & -1 & 1 & 1 & -1 & -1 & 1 & 1 & -1 & -1 & 1 \\ \hline
	5	& -1 & 1 & -1 & -1 & 1 & 1 & -1 & -1 & 1 & 1 & -1 & -1 & -1 & -1 & 1 \\ \hline
	6	& 1 & -1 & 1 & 1 & 1 & -1 & -1 & -1 & -1 & 1 & -1 & 1 & -1 & -1 & 1 \\ \hline
	7	& -1 & -1 & -1 & 1 & 1 & 1 & 1 & -1 & 1 & -1 & 1 & 1 & -1 & -1 & 1 \\ \hline
	8	& 1 & 1 & -1 & 1 & -1 & 1 & 1 & -1 & -1 & 1 & -1 & -1 & -1 & 1 & 1 \\ \hline
	9	& -1 & 1 & 1 & 1 & -1 & -1 & -1 & -1 & 1 & -1 & 1 & -1 & -1 & 1 & 1 \\ \hline
\end{tabular}}
	\end{table}  
\end{frame}



\begin{frame}
  \frametitle{\titleE}
	\begin{table}[ht!]
		\centering
		\scalebox{0.7}{
		\begin{tabular}{|l|l|l|l|l|l|l|l|l|l|l|l|} \hline
			\multicolumn{1}{|p{1.8cm}|}{\vspace*{.7 cm}\# training points} &
			\multicolumn{1}{p{1.8cm}|}{\vspace*{0 cm}\hbox{One-vs-All} uniquely classfied, linear} &
			\multicolumn{1}{p{1.8cm}|}{\vspace*{0 cm}\hbox{One-vs-All} with bary- centers, linear} &
			\multicolumn{1}{p{1.8cm}|}{\vspace*{0 cm}\hbox{One-vs-All} uniquely classfied, Gaussian} &
			\multicolumn{1}{p{1.8cm}|}{\vspace*{0 cm}\hbox{One-vs-All} with bary-centers, Gaussian} &
			\multicolumn{1}{p{1.8cm}|}{\vspace*{.7 cm}ECOC, \hbox{linear}} &
			\multicolumn{1}{p{1.8cm}|}{\vspace*{.7 cm}ECOC, Gaussian} \\ \hline \hline
			500	& \visible<2->{65.9\%} & \visible<2->{74.1\%} & \visible<2->{75.4\%} & \visible<2->{83.3\%} & \visible<2->{74.2\%} & \visible<2->{87.4\%} \\ \hline
1000	& \visible<2->{68.2\%} & \visible<2->{75.0\%} & \visible<2->{84.3\%} & \visible<2->{89.0\%} & \visible<2->{78.0\%} & \visible<2->{92.7\%} \\ \hline
2000	& \visible<2->{70.2\%} & \visible<2->{76.4\%} & \visible<2->{89.8\%} & \visible<2->{91.9\%} & \visible<2->{77.8\%} & \visible<2->{94.3\%} \\ \hline
5000	& \visible<2->{70.0\%} & \visible<2->{73.8\%} & \visible<2->{88.9\%} & \visible<2->{91.6\%} & \visible<2->{82.0\%} & \visible<2->{95.2\%} \\ \hline
10000	& \visible<2->{64.6\%} & \visible<2->{67.5\%} & \visible<2->{88.0\%} & \visible<2->{90.6\%} & \visible<2->{82.5\%} & \visible<2->{95.4\%} \\ \hline
		\end{tabular}
		}
		\caption{Correctly Classified Digits}
	\end{table}
\end{frame}

\begin{frame}
  \frametitle{\titleE}
	\begin{figure}[h]
		\includegraphics[width=1\textwidth]{mnist_really_bad_images}
		\caption{Visualizing very illegible digits}
	\end{figure}
\end{frame}

\end{document}

