Support Vector Machines are binary classifiers, meaning that there only two possible output labels. Usually, one chooses them to be $1$ and $-1$. Our problem, however, was to classify unknown data into one out of ten different categories. Since rewriting the SVM algorithm to produce a separation into multiple classes is linked to great effort, we opt for finding a way to combine several SVMs, which together would be able to choose from more than one label. There are a number of well known strategies tackling this issue. Let us first look at two very intuitive but also primitive strategies. The first one is known as One-vs-All classification. In this approach one trains as many SVMs as there are classes, i.e. ten in our case, and sets labels so that the $i$-th SVM has all training data with label $i$ set to 1 and everything else to $-1$. 

....

This approach has many issues. Not uniquely classified. Advantage low compute time, one has to at least expect to train $10$ binary classifiers to divide into $10$ different categories. depends on which one first...? some areas several labels...  

The other possibility is called One-vs-One classification. Here one only ever considers the data corresponding to two classes and sets one label to $1$ and the other one to $-1$. ... 
In our case this would mean training $45$ SVMs (the number arises from the binary coefficient of 10 choose 2...). advantage/disadvantage

We only tried the first approach since as just mentioned the compute time for the second one was simply too high for us considering that we have a very large amount of training data. In the One-vs-All results were very very disappointing. In the case of a linear as well as a Gaussian kernel ... In the linear case this can be explained by the fact that our training data was simply not linearly seperable. 
Hence we added in an additional feature. We computed the barycenters of the data points of each class. When a data point could not be uniquely classified we would give it the label of the barycenter it was closest to. This method led to a major improvement of our results. Our algorithm now labeled about 60\% of our data correctly (... see appendix).  Nevertheless the result was still not satisfactory. 

So we decided to focus our attention on a different approach: Error Correcting Output Codes. 
\\
subheading? \\
\\
They are easiest to explain considering a concrete example. We trained 15 different classifiers $f_i$. 

\begin{table}[ht!]
	\centering
	\caption{Error Correcting Output Codes}
	\label{Codewords}
	\begin{tabular}{|l|l|l|l|l|l|l|l|l|l|l|l|l|l|l|l|}
	\hline
	Class	& $f_0$ & $f_1$ & $f_2$ & $f_3$ & $f_4$ & $f_5$ & $f_6$ & $f_7$ & $f_8$ & $f_9$ & $f_{10}$ & $f_{11}$ & $f_{12}$ & $f_{13}$ & $f_{14}$ \\ \hline \hline
	0	& 1 & 1 & -1 & -1 & -1 & -1 & 1 & -1 & 1 & -1 & -1 & 1 & 1 & -1 & 1 \\ \hline
	1	&  &  &  &  &  &  &  &  &  &  &  &  &  &  & \\ \hline
	2	&  &  &  &  &  &  &  &  &  &  &  &  &  &  & \\ \hline
	3	&  &  &  &  &  &  &  &  &  &  &  &  &  &  & \\ \hline
	4	&  &  &  &  &  &  &  &  &  &  &  &  &  &  & \\ \hline
	5	&  &  &  &  &  &  &  &  &  &  &  &  &  &  & \\ \hline
	6	&  &  &  &  &  &  &  &  &  &  &  &  &  &  & \\ \hline
	7	&  &  &  &  &  &  &  &  &  &  &  &  &  &  & \\ \hline
	8	&  &  &  &  &  &  &  &  &  &  &  &  &  &  & \\ \hline
	9	&  &  &  &  &  &  &  &  &  &  &  &  &  &  & \\ \hline
	\end{tabular}
\end{table}  

The rows of the table show what class is assigned what label depending on each classifier. For example $f_0$ assigns $1$'s to all even numbers and $-1$'s to all odd numbers. This way each class has a string of $1$'s and $-1$'s it corresponds to which is also called its codeword. The codewords are chosen such that their Hamming distance (i.e. the number of entries where they differ) is maximized. In our case each codeword has a Hamming distance of at least six to any of the others strings. Now when one wants to label an unknown data point each of the classifiers assigns it a label and one gets an output string of 15 digits. We classify the data point according the codeword it has the least Hamming distance to. The name error correcting output codes is derived from the fact that we can for example three classifiers can misclassify a data point and we will still get the correct result. Depending on the difficulty of the problem one could choose more or less classifiers and hece in-or decrease the Hamming distance of the set of codewords. We took these codewords from ... and yielded much better results than with the approaches mentioned above. Our precise implementation can be found on page... in the appendix. 

When choosing ... many training points and .... many testing points our algorithm was correct in ... of the cases. Include things here...?

Linear and Gaussian.

In general ECOC?

Where do we bring in cross validation?

%Es wurden auch einige Eigenschaften der Fatou-Menge beleuchtet, unser Hauptaugenmerk lag jedoch auf der Julia-Menge.Trotzdem die Iterierten auf der Fatou-Menge normal sind und die Dynamik der Funktion hier in dem Sinne gut zu beschreiben ist, dass benachbarte Punkte unter Iteration ein ähnliches Verhalten zeigen, hat sie doch viele interessante Eigenschaften. Beispielsweise kann man die Zusammenhangskomponenten der Fatou-Menge klassifizieren und wird feststellen, dass es recht wenige Typen gibt. Mehr dazu findet sich in \cite{Schleicher}.