\documentclass[12pt, a4paper, openany]{report} %, BCOR0mm scrbook

% scrartcl ist eine abgeleitete Artikel-Klasse im Koma-Skript
% bibliography=totoc: Literaturverzeichnis im Inhaltsverzeichnis
% zur Kontrolle des Umbruchs Klassenoption draft verwenden
%BCOR: Bindekorrektur


% die folgenden Packete erlauben den Gebrauch von Umlauten und ß
% in der Latex Datei
\usepackage[utf8]{inputenc}
% \usepackage[latin1]{inputenc} %  Alternativ unter Windows
\usepackage[T1]{fontenc}
\usepackage[ngerman]{babel}


\usepackage[paper=a4paper,left=25mm,right=25mm,top=25mm,bottom=25mm]{geometry}
\usepackage[pdftex]{graphicx}
\usepackage{subfigure}
\usepackage{latexsym}
\usepackage{amsmath,amssymb,amsthm}
\usepackage[export]{adjustbox}
\usepackage{pdfpages}
% deutsches Literaturverzeichnis
\usepackage[numbers,round]{natbib}

%neue Seite für jedes Kapitel
%\usepackage{titlesec}
%\newcommand{\sectionbreak}{\clearpage}

% Umgebungen für Definitionen, Sätze, usw.
% Es werden Sätze, Definitionen etc innerhalb eines Chapters mit
% 1.1, 1.2 etc durchnummeriert, ebenso die Gleichungen mit (1.1), (1.2) ..
\newtheorem{Satz}{Satz}[chapter]
\newtheorem{Lemma}[Satz]{Lemma}
\newtheorem{Korollar}[Satz]{Korollar}

\theoremstyle{definition}	
\newtheorem{Definition}[Satz]{Definition}
\newtheorem{Beispiel}[Satz]{Beispiel}
\newtheorem{Bemerkung}[Satz]{Bemerkung}
                    
\numberwithin{equation}{chapter} 

% einige Abkuerzungen
\newcommand{\C}{\mathbb{C}} % komplexe
\newcommand{\K}{\mathbb{K}} % komplexe
\newcommand{\R}{\mathbb{R}} % reelle
\newcommand{\Q}{\mathbb{Q}} % rationale
\newcommand{\Z}{\mathbb{Z}} % ganze
\newcommand{\N}{\mathbb{N}} % natuerliche
\DeclareMathOperator{\sign}{sign}



\begin{document}
  % Keine Seitenzahlen im Vorspann
  \pagestyle{empty}
\begin{titlepage}
	
	\newcommand{\HRule}{\rule{\linewidth}{0.5mm}} % Defines a new command for the horizontal lines, change thickness here
	
	\center % Center everything on the page
	
	%----------------------------------------------------------------------------------------
	%	HEADING SECTIONS
	%----------------------------------------------------------------------------------------
	\hspace*{5.0cm}\\[4cm]

	\textsc{\LARGE Mathematical Aspects\\[0.2cm] of Machine Learning}\\[0.5cm] % Major heading such as course name
	\textsc{\large Report}\\[2cm] % Minor heading such as course title
	
	%----------------------------------------------------------------------------------------
	%	TITLE SECTION
	%----------------------------------------------------------------------------------------
	
	\HRule \\[0.8cm]
	{ \huge \bfseries Digit Recognition\\ with Support Vector Machine}\\[0.8cm] % Title of your document
	\HRule \\[1.5cm]
	
	%----------------------------------------------------------------------------------------
	%	AUTHOR SECTION
	%----------------------------------------------------------------------------------------
	
	\begin{minipage}{0.4\textwidth}
		\begin{flushleft} \large
			\emph{Authors:}\\
			Lisa \textsc{Gaedke-Merzhäuser} \\% Your name
			Paul \textsc{Korsmeier}\\
			Lisa \textsc{Mattrisch}\\
			Vanessa \textsc{Schreck}\\
		\end{flushleft}
	\end{minipage}
	~
	\begin{minipage}{0.4\textwidth}
		\begin{flushright} \large
			\emph{} \\
			 \textsc{} % Supervisor's Name
		\end{flushright}
	\end{minipage}\\[2cm]

	
	
\end{titlepage}

\cleardoublepage
  % Ab sofort Seitenzahlen unten mittig
  \pagestyle{plain}

\chapter*{Problem Statement}
The main goal of our project was to correctly identify handwritten digits based on the MNIST  ("Modified National Institute of Standards and Technology") data set.
This data set consists of 42,000 gray-scale images. Each image is 28 pixels in height and 28 pixels in width. Each pixel has a single pixel-value associated with it, indicating the lightness or darkness of that pixel \cite{kaggel}.

\begin{figure}[h]
	\includegraphics[width=1\textwidth, center]{Digits2}
	\caption{Visualization of eight of these images}
\end{figure}


Formally speaking, we have points  $\{x_i\}_{i=1}^l \subset \R^{28^2}$ with respective labels from $\{0,1,\ldots, 9\}$. Our goal is to now find a classification function $f$ that gives for each of these points a prediction $f(x) \in \{0,1,\ldots, 9\}$ which ideally matches the actual label.\\

There are several ways to achieve this goal, one of which is the concept of {kernelized} support vector machines (SVM). The main idea behind this algorithm is to not only separate the data according to their labels, but also to do so in a way that is as reasonable as possible, i.e. we want to maximize the distance of the data to the decision boundary. We first transfer the data with a feature map $\Phi$ into a suitable feature space and look for a linear separation there.\\

From this notion, two particular problems arise. Firstly how to solve the optimization problem of maximizing the margin (cf. Section 1.1 and 2) and secondly: The SVM is a binary classifier, i.e. it can only handle data that is divided into two classes. We therefore decided to compare different ways to deal with this obstacle, generalizing the application of SVMs to multiclass  classification problems (cf. Section \ref{ch_multiclass}). The methods that we considered all combine several instances of the SVM in its original form, classifying according to labels $\{\pm 1\}$.\\

We can summarize this binary classification problem in the objective of finding a decision function of the form
\begin{equation*}
f(x) = \sign \left(w^T\Phi(x) + b\right).
\end{equation*}


\subsection{Binary Support Vector Machine}


Approaching binary support vector machines from a more technical side, we find that they aim to find the maximum margin hyperplane in the feature space, i.e. the goal is to maximize the margin while softly penalizing points that lie on the wrong side of the boundary or inside the margin \cite{Bishop2006}. Dualizing this optimization problem, we obtain the following equivalent quadratic program:

\begin{eqnarray}\label{eq:dual_problem}
	\text{minimize} & d(\alpha) := \frac{1}{2} \alpha^T Q \alpha - 1^T \alpha  \\ 
	\nonumber
	\text{s.t.} &  0 \leq \alpha \leq C  \quad \text{and} \quad 	y^T \alpha = 0,
\end{eqnarray}

where $\alpha$ is the dual variable, $Q_{ij} = y_i y_j k(x_i,x_j)$, $k$ the kernel function and $C = 1/(2 \lambda l)$ for the penalty term $\lambda$ of the soft margin SVM.
After finding the optimal solution $\alpha^*$ of this program, we can formulate the resulting decision function as
\begin{equation*}
f(x) = \operatorname{sgn}\left(\sum_{i=1}^{l}\alpha^*_i y_i k(x_i,x_j)+b\right).
\end{equation*}
Note that only data points $x_i$ with $\alpha_i \neq 0$ appear in the decision function. Those points are called \textit{support vectors}.\\
By analyzing primal, dual and the corresponding KKT conditions, we find that any $x_i$ with $\alpha_i = C$ was correctly classified and lies exactly on the margin boundary, and any $x_i$ with $0<\alpha_i < C$ lies either inside the margin or on the wrong side of the margin. Additionally, any data points $x_i$ on the (correct) margin boundary must satisfy $y_i f(x_i) = 1$. We can use this property to find the parameter $b$ \cite{Bishop2006}. Hence, if we succeed to find an optimal solution to \eqref{eq:dual_problem}, we have everything we need to construct the decision function.





\chapter*{Support Vector Machine}
Given data points $\{x_i\}_{i=1}^l \subset \R^d$ with respective labels $\{y_i\}_{i=1}^l \subset \{\pm 1\}$ we want to seperate differently labelled points using a decision function of the form
\begin{equation*}
f(x) = \operatorname{sgn}\left(w^T\Phi(x) + b\right)
\end{equation*}
where $\Phi$ is the feature map as introduced in the lecture.

The Support Vector Machine (SVM) is a binary classifier that aims to find the maximum margin hyperplane in the feature space, i.e. the goal is to maximize the margin while softly penalizing points that lie on the wrong side of the boundary or inside the margin \cite{Bishop2006}. Dualizing this optimization problem, we obtain the following equivalent quadratic program:
\begin{equation*}
\begin{aligned}
& \underset{0 \leq \alpha \leq C}{\text{maximize}}
& & \sum_{i = 1}^{l}\alpha_i - \frac{1}{2} \sum_{i, j = 1}^{l}\alpha_i  \alpha_j y_i y_j k(x_i, x_j)
\end{aligned}
\end{equation*}
with the dual variable $\alpha$, kernel function $k$ and penalty $C$.
After finding the optimal solution $\alpha^*$ of this program, we can formulate the resulting decision function as
\begin{equation*}
f(x) = \operatorname{sgn}\left(\sum_{i=1}^{l}\alpha^*_i y_i k(x_i,x_j)+b\right).
\end{equation*}
Note that only data points $x_i$ with $\alpha_i \neq 0$ appear in the decision function. Those points are called \textit{support vectors}. By analysing primal, dual and the corresponding KKT conditions, we find that any $x_i$ with $\alpha_i = C$ was correctly classified and lies exactly on the margin boundary, and any $x_i$ with $0<\alpha_i < C$ lies either inside the margin or on the wrong side of the margin. Additionally, any data points $x_i$ on the (correct) margin boundary must satisfy $y_i\cdot f(x_i) = 1$. We can use this property to find the parameter $b$ \cite{Bishop2006}.



\chapter*{Solving the optimization problem with SMO}
\subsection{Some theory on SMO convergence}
Let $\alpha$ be some feasible variable for Problem (\ref{eq:dual_problem}). Defining
\begin{align*}
\mathcal{L}(\alpha,\delta,\mu,\beta) &:= \frac{1}{2} \alpha^T Q \alpha - 1^T \alpha - \delta^T\alpha + \mu^T(\alpha - C) - \beta \alpha^T y \\
F_i(\alpha) &:= y_i (\partial_i d)(\alpha) = \sum_{i = 1}^l \alpha_j y_j k(x_i,x_j) - y_i \quad \text{for} \quad i = 1,\ldots,l,
\end{align*}
by careful manipulations we find that the KKT optimality conditions
\begin{equation*}
\left.
\begin{aligned}
\nabla_{\alpha} \mathcal{L}(\alpha^*,\delta^*,\mu^*,\beta^*) &= 0\\
\delta_i^* &\geq 0\\
\delta_i^* \alpha_i^* &= 0\\
\mu_i^* &\geq 0\\
\mu_i^* (\alpha_i^*-C) &= 0\\
\alpha_i^* &\text{ feasible}
\end{aligned}
\right\} \text{for all } i \in \{1,\ldots,l\}
\end{equation*}
for a solution of Problem (\ref{eq:dual_problem}) (which are sufficient, since Q is spsd), are equivalent to the -- much simpler looking -- pairwise condition
\begin{equation*}\label{equiv_KKT}
b_{up}(\alpha) := \min_{i \in I_{up}(\alpha)} F_i(\alpha) \geq \max_{j \in I_{low}(\alpha)} F_j(\alpha) =: b_{low}(\alpha),
\end{equation*}
where $I_{up}(\alpha)$ and $I_{low}(\alpha)$ are subsets of the index set $\{1,\ldots,l\}$ defined by
\begin{align*}
I_{up}(\alpha) &:= \{ i  \mid  \alpha_i < C \text{ and } y_i = 1 \text{ or } \alpha_i > 0 \text{ and } y_i = -1 \} \\
I_{low}(\alpha) &:= \{ j \mid \alpha_j < C \text{ and } y_j = -1 \text{ or } \alpha_j > 0 \text{ and } y_j = 1 \}.
\end{align*} 
Any pair $(i,j) \in I_{up}(\alpha) \times I_{low}(\alpha)$ with $F_i(\alpha) < F_j(\alpha)$ is thus called a \textit{violating pair} and an objective equivalent to solving Problem (\ref{eq:dual_problem}) is to change $\alpha$ so as to remove all such violating pairs. Since a priori we do not know if the solution $\alpha^*$ fulfils (\ref{equiv_KKT}) strictly or not, we define, for some small tolerance $\tau > 0$, a $\tau$-violating pair as some $(i,j) \in I_{up}(\alpha) \times I_{low}(\alpha)$ which satisfies $F_i(\alpha) < F_j(\alpha) - \tau$ and require that all $\tau$-violating pairs be removed, or equivalently, 
\begin{equation*}\label{tau_KKT}
b_{up}(\alpha) \geq b_{low}(\alpha) - \tau.
\end{equation*}
It holds (see \cite{Keerthi2002} for a proof) that any algorithm of the following form terminates after finitely many steps:
\begin{algorithm}[General SMO type algorithm]\label{GSMO} Let $\tau > 0$. Initialize $k = 0 $ and $\alpha^0 = 0$ and generate iterates $\alpha^k$, $k \in \mathbb{N},$ as follows: 
\begin{enumerate}
\item If $\alpha^k$ satisfies (\ref{tau_KKT}), stop. Else choose a $\tau$-violating pair $(i,j) \in I_{up}(\alpha^k) \times I_{low}(\alpha^k)$.
\item Minimize $d$ varying only $\alpha^k_i$ and $\alpha^k_j$, leaving $\alpha^k_n$ fixed for $n \notin \{i,j\}$ and respecting the constraints of Problem (\ref{eq:dual_problem}) to obtain $\alpha^{\text{new}}$.
\item Set $k := k+1$, $\alpha^k := \alpha^{\text{new}}$ and go to Step 1.
\end{enumerate}
\end{algorithm} 
Platt's original Sequential Minimal Optimization (SMO) algorithm (actually, we are referring to "Modification 1" by \cite{KeerthiShevade}), although ground-breaking, has two weaknesses. Firstly, its pseudocode description is quite complex, making it hard to judge whether or not one has implemented it as intended by its originators. Secondly, by trying to avoid computational effort in the \textit{while} steps, it actually runs much longer than all other algorithms we have implemented or tested.\\\\
In short, Platt's SMO tries first to ensure that
\[
b_{up,I_0}(\alpha) := \max_{i \in I_0(\alpha)} F_i(\alpha) \geq \min_{j \in I_0} F_j(\alpha) =: b_{low,I_0}(\alpha),
\]
where $I_0(\alpha) := \{ i  \mid  0 < \alpha_i < C \}$ is the index set of ``interior'' $\alpha_i$s. A cache of only the $F_i$ for $i \in I_0$ is kept until this is achieved. Then all $i \in \{0,\ldots l\}$ are examined, and the cache of correctly stored $F_i$s (along with the indices $\widetilde{i_{up}}$ and $\widetilde{i_{low}}$ indicating where the so far most extreme $F_i$ occur) is extended as long as an $\alpha_j$ is found such that $(j,\widetilde{i_{low}})$ or $(\widetilde{i_{up}},j)$ is violating, depending on whether $j \in I_{up}$ or $j \in I_{low}$.\\\\
Therefore, the algorithm does not resolve a \textit{maximally} violating pair, by which we mean some 
\[
(i_{up},i_{low}) \in \argmin_{i \in I_{up}(\alpha)}F_i(\alpha) \times \argmax_{j \in I_{low}(\alpha)}F_j(\alpha).
\] To look up such a pair is the strategy of the ``Working Set Selection 1'' (WSS1) approach in \cite{FanChenLin} and a rewarding investment, see benchmarking section. \\\\
Parametrize the 1D Problem in Step 2 of Algorithm \ref{GSMO} as
\[
\alpha_i(t^*) = \alpha^k_i + y_i t*, \quad \alpha_j(t^*) = \alpha^k_j - y_j t*, \quad \alpha_n(t^*) = \alpha^k_n \text{ for } k \notin \{i,j\}
\]
for some optimal feasible $t^* \in \R$ and set $\Phi(t) = d(\alpha(t))$ as the 1D objective function. Then actually
\begin{equation*}
F_i(\alpha) - F_j(\alpha) = \Phi'(0),
\end{equation*}
so WSS1 corresponds to a steepest descent approach aiming for a substantial decrease in $d$. Since this uses only first order information, a second ``Working Set Selection 2'' (WSS2) strategy is proposed in \cite{FanChenLin}, which (at relatively low extra cost) employs second order information to choose a violating pair almost optimally: $i \in \argmin_{i \in I_{up}(\alpha)}F_i(\alpha)$ and $j \in I_{low}$ such that the decrease in $d$ produced by removing the violating pair is maximal.

\subsection{Implementational issues}
One particular source of trouble was that WSS1 and WSS2 sometimes got stuck on a single violating pair, which is extremely puzzling because -- in theory -- a violating pair $(i,j)$ of indices can no longer be violating after an SMO step has been taken. It turned out that the precise definitions of $I_{up}$ and $I_{low}$ had to be softened so that numerical errors do not stop the algorithm from recognizing that $(\alpha^{new}_i,\alpha^{new}_j$ has hit the boundary of $[0,C]^2$. This is crucial for steps where the box constraints are active for $(\alpha^{new}_i,\alpha^{new}_j)$. For further details, please see the commented code we provided. 
\subsection{Benchmarking different SMOs}

We implemented Platt's SMO, WSS1 in two variations and WSS2. The first variation of WSS1 caches all computations of Gramian matrix rows in order to avoid expensive double computations. The other variant of WSS1 is forgetful. Taking the scikit-learn SVC algorithm as a 5th (tough) competitor, we tested the algorithms at hand for the first classifier of ECOC for a range of numbers of MNIST training points. \\\\
As can be seen, the speediest algorithms are, in decreasing order: scikit-learn, WSS1 with caching, WSS2 with caching, WSS1 without caching, Platt's SMO. Somewhat surprisingly, WSS2 with caching is not the fastest method, although in theory it should be superior to WSS1 with caching. Apparently, the extra computational cost outweighs the theoretical gain. We suppose that the strange kink in the WSS1 with caching graph (Figure \ref{bench_ssp}) is due to multicore features of Python kicking in at 1800 training points. The benchmarks were carried out on a Windows 7 x64 machine with an Intel i5 quad core processor at 4 GHz and 16 GB of RAM.\\\\
The polynomial order of all algorithms is 2, judging from the loglog plot in Figure \ref{bench_gauss_loglog} and a linear least squares fit of the loglog data for WSS1 with caching. The latter also yielded that our WSS1 with caching takes about 5.8 times as long as scikit-learn.

\begin{figure}[h!]
	\includegraphics[width=0.7\textwidth, center]{benchplot_gauss.pdf}
	\caption{Run times for different SMO algorithms, Gaussian kernel}
	\label{bench_gauss}
\end{figure}

\begin{figure}[h!]
	\includegraphics[width=0.7\textwidth, center]{benchplot_ssp.pdf}
	\caption{Run times for different SMO algorithms, standard scalar product}
	\label{bench_ssp}
\end{figure}

\begin{figure}[h!]
	\includegraphics[width=0.7\textwidth, center]{benchplot_gauss_loglog.pdf}
	\caption{loglog plot of run times, Gaussian kernel}
	\label{bench_gauss_loglog}
\end{figure}





\newpage


\chapter*{Multi Class Classification Problem}
Support Vector Machines are binary classifiers, meaning that there only two possible output labels. Usually, one chooses them to be $1$ and $-1$. Our problem, however, was to classify unknown data into one out of ten different categories. Since rewriting the SVM algorithm to produce a separation into multiple classes is linked to great effort, we opt for finding a way to combine several SVMs, which together would be able to choose from more than one label. There are a number of well known strategies tackling this issue. Let us first look at two very intuitive but also primitive strategies. The first one is known as One-vs-All classification. In this approach one trains as many SVMs as there are classes, i.e. ten in our case, and sets labels so that the $i$-th SVM has all training data with label $i$ set to 1 and everything else to $-1$. 

....

This approach has many issues. Not uniquely classified. Advantage low compute time, one has to at least expect to train $10$ binary classifiers to divide into $10$ different categories. depends on which one first...? some areas several labels...  

The other possibility is called One-vs-One classification. Here one only ever considers the data corresponding to two classes and sets one label to $1$ and the other one to $-1$. ... 
In our case this would mean training $45$ SVMs (the number arises from the binary coefficient of 10 choose 2...). advantage/disadvantage

We only tried the first approach since as just mentioned the compute time for the second one was simply too high for us considering that we have a very large amount of training data. In the One-vs-All results were very very disappointing. In the case of a linear as well as a Gaussian kernel ... In the linear case this can be explained by the fact that our training data was simply not linearly seperable. 
Hence we added in an additional feature. We computed the barycenters of the data points of each class. When a data point could not be uniquely classified we would give it the label of the barycenter it was closest to. This method led to a major improvement of our results. Our algorithm now labeled about 60\% of our data correctly (... see appendix).  Nevertheless the result was still not satisfactory. 

So we decided to focus our attention on a different approach: Error Correcting Output Codes. 
\\
subheading? \\
\\
They are easiest to explain considering a concrete example. We trained 15 different classifiers $f_i$. 

\begin{table}[ht!]
	\centering
	\caption{Error Correcting Output Codes}
	\label{Codewords}
	\begin{tabular}{|l|l|l|l|l|l|l|l|l|l|l|l|l|l|l|l|}
	\hline
	Class	& $f_0$ & $f_1$ & $f_2$ & $f_3$ & $f_4$ & $f_5$ & $f_6$ & $f_7$ & $f_8$ & $f_9$ & $f_{10}$ & $f_{11}$ & $f_{12}$ & $f_{13}$ & $f_{14}$ \\ \hline \hline
	0	& 1 & 1 & -1 & -1 & -1 & -1 & 1 & -1 & 1 & -1 & -1 & 1 & 1 & -1 & 1 \\ \hline
	1	&  &  &  &  &  &  &  &  &  &  &  &  &  &  & \\ \hline
	2	&  &  &  &  &  &  &  &  &  &  &  &  &  &  & \\ \hline
	3	&  &  &  &  &  &  &  &  &  &  &  &  &  &  & \\ \hline
	4	&  &  &  &  &  &  &  &  &  &  &  &  &  &  & \\ \hline
	5	&  &  &  &  &  &  &  &  &  &  &  &  &  &  & \\ \hline
	6	&  &  &  &  &  &  &  &  &  &  &  &  &  &  & \\ \hline
	7	&  &  &  &  &  &  &  &  &  &  &  &  &  &  & \\ \hline
	8	&  &  &  &  &  &  &  &  &  &  &  &  &  &  & \\ \hline
	9	&  &  &  &  &  &  &  &  &  &  &  &  &  &  & \\ \hline
	\end{tabular}
\end{table}  

The rows of the table show what class is assigned what label depending on each classifier. For example $f_0$ assigns $1$'s to all even numbers and $-1$'s to all odd numbers. This way each class has a string of $1$'s and $-1$'s it corresponds to which is also called its codeword. The codewords are chosen such that their Hamming distance (i.e. the number of entries where they differ) is maximized. In our case each codeword has a Hamming distance of at least six to any of the others strings. Now when one wants to label an unknown data point each of the classifiers assigns it a label and one gets an output string of 15 digits. We classify the data point according the codeword it has the least Hamming distance to. The name error correcting output codes is derived from the fact that we can for example three classifiers can misclassify a data point and we will still get the correct result. Depending on the difficulty of the problem one could choose more or less classifiers and hece in-or decrease the Hamming distance of the set of codewords. We took these codewords from ... and yielded much better results than with the approaches mentioned above. Our precise implementation can be found on page... in the appendix. 

When choosing ... many training points and .... many testing points our algorithm was correct in ... of the cases. Include things here...?

Linear and Gaussian.

In general ECOC?

Where do we bring in cross validation?

%Es wurden auch einige Eigenschaften der Fatou-Menge beleuchtet, unser Hauptaugenmerk lag jedoch auf der Julia-Menge.Trotzdem die Iterierten auf der Fatou-Menge normal sind und die Dynamik der Funktion hier in dem Sinne gut zu beschreiben ist, dass benachbarte Punkte unter Iteration ein ähnliches Verhalten zeigen, hat sie doch viele interessante Eigenschaften. Beispielsweise kann man die Zusammenhangskomponenten der Fatou-Menge klassifizieren und wird feststellen, dass es recht wenige Typen gibt. Mehr dazu findet sich in \cite{Schleicher}.



\bibliography{Report_bib}
  \bibliographystyle{alphadin}

%\cleardoublepage
%\includepdf{Selbstaendigkeitserklaerung.pdf}  



\end{document}

